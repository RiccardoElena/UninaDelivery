\chapter{Analisi dei requisiti e progettazione del sistema}

\section{Introduzione e Requisiti}

È richiesto di progettare un \textbf{applicativo a interfaccia grafica} per la gestione di un \textbf{azienda 
di delivery} dal lato degli operatori di tale ditta.


Una volta effettuato l'accesso, l'operatore deve poter:

\begin{itemize}
  \item Visualizzare la lista degli ordini in attesa di essere evasi
  \item Creare spedizioni per tali ordini
  \item Visualizzare report mensili sugli ordini evasi
\end{itemize}

\section{Assunzioni sul Dominio}

Si assume che l'azienda di delivery spedisca \textbf{merce propria} o \textbf{di terzi} lasciata in 
gestione. Si assume inoltre che l'azienda si occupi di spedizioni \textbf{nazionali e 
internazionali} e che l'operatore gestisca, oltre che le spedizioni \textbf{verso i clienti},
anche gli \textbf{spostamenti di merci} da un deposito all'altro e le \textbf{spedizioni 
intermedie}\footnote{L'applicativo è predisposto all'implementazione di queste 
feature che però non è effettivamente implementata, essendo compresa nelle 
richieste extra non assegnateci.}.

Inoltre, per ragioni di gestione della \textbf{capillarità del servizio}, si assume che le
spedizioni \textbf{dirette verso un cliente} partano esclusivamente da depositi \textbf{della stessa
città} del cliente e che, in caso i prodotti richiesti non siano presenti, sia necessario
uno spostamento di merci o una spedizione intermedia prima di concludere tale spedizione.

Infine, per ragioni di gestione del sistema, si assume che ogni ordine \textbf{possa contenere
un solo prodotto}, con quantità variabile, e che quindi ordini contenenti più prodotti
vengano, una volta richiesti dal cliente, suddivisi \textbf{in più ordini prima di essere inseriti}
dall'applicativo destinato ai clienti\footnote{Anche per questa feature l'applicativo è 
predisposto all'implementazione, pur non essendo effettivamente presente per ragioni simili 
alle precedenti.}. Una spedizione \textbf{può contenere più ordini}, a patto che siano tutti destinati
alla \textbf{stessa zona di destinazione}, identificata dal CAP, nel caso di spedizione verso clienti,
o lo stesso \textbf{deposito di arrivo}, in caso di spedizioni intermedie o spostamento merci.



