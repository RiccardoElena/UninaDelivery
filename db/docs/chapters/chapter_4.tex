\chapter{Implementazione del Database}

In conclusione si riportano le istruzioni per la creazione del database e la sua popolazione.

Come DBMS provider si è scelto di utilizzare PostgreSQL, per la sua facilità di utilizzo e la sua affidabilità.

Il database è hostato localmente per rendere più agevole la connessione.

Per pulizia del codice si è preferito lavorare su uno \textbf{schema} creato appositamente per il progetto, chiamato \textbf{UninaDelivery}.

\section{Definizione dei Domini Personalizzati}

Vista la vasta gamma di vincoli di dominio simili su vari campi, si è deciso di definire dei domini personalizzati per facilitare la creazione delle tabelle.

\begin{lstlisting}
  CREATE DOMAIN letterString AS text CHECK (VALUE ~ '^[a-zA-Z]+$');
  
  CREATE DOMAIN numberString AS text CHECK (VALUE ~ '^[0-9]+$');
  
  CREATE DOMAIN alphanumericString AS text CHECK (VALUE ~ '^\w+$');
  
  CREATE DOMAIN emailString AS text CHECK (
    VALUE ~ '^\w+[\w.]*\w@[a-zA-Z.]+\.[a-zA-Z]{2,}$'
  );
\end{lstlisting}

Inoltre per i campi che nel diagramma ristrutturato sono stati definiti come \textbf{enumerazioni}, si è deciso di utilizzare dei tipi enumerati personalizzati.

\begin{lstlisting}
  CREATE TYPE WorldZone AS ENUM (
    'NA', 'EUW', 'EUNE', 'LATAM', 'MIDEAST', 'CKJ', 'SEA', 'IND', 'RUS', 'STAN', 
    'OC', 'AFN', 'AFC', 'AFS'
  );

  CREATE TYPE DepositType AS ENUM (
    'City', 'State', 'Country', 'Central'
  );

  CREATE TYPE TransportType AS ENUM (
    'WheeledSmall', 'WheeledLarge', 'Rails', 'Water', 'Air'
  );
  
  CREATE TYPE DrivingLicenceType AS ENUM ('BE', 'CE');
\end{lstlisting}

\section{Creazione delle Tabelle}

Vediamo ora la creazione delle tabelle, con annessi vincoli di dominio e di n-upla, oltre che le chiavi primarie e le chiavi esterne.

\begin{lstlisting}[caption={Creazione della tabella \textbf{Area}}]
  CREATE TABLE Area (
    ZipCode numericString,
    City letterString,
    State letterString,
    Country letterString,
    WorldZone WorldZone
  );
  ALTER TABLE Area ADD CONSTRAINT Area_pk PRIMARY KEY (ZipCode, Country);
\end{lstlisting}

\begin{lstlisting}[caption={Creazione della tabella \textbf{Address}}]
  CREATE TABLE Address (
    AddressNo alphNumString NOT NULL,
    Street alphNumString NOT NULL,
    ZipCode numericString NOT NULL,
    Country letterString NOT NULL
  );
  ALTER TABLE Address ADD CONSTRAINT Address_pk PRIMARY KEY (
    AddressNo,Street, ZipCode, Country
  );
  ALTER TABLE Address ADD CONSTRAINT AddressNo_domain CHECK (
    AddressNo ~ '^[1-9]\d*[a-z]?(?:BIS)?$'
  );
  ALTER TABLE Address ADD CONSTRAINT Address_fk FOREIGN KEY 
    (ZipCode,Country) REFERENCES Area(ZipCode,Country) 
    ON DELETE CASCADE;
\end{lstlisting}
    