\chapter{Implementazione del Database}

In conclusione si riportano le istruzioni per la creazione del database e la sua popolazione.

Come DBMS provider si è scelto di utilizzare PostgreSQL, per la sua facilità di utilizzo e la sua affidabilità.

Il database è hostato localmente per rendere più agevole la connessione.

Per pulizia del codice si è preferito lavorare su uno \textbf{schema} creato appositamente per il progetto, chiamato \textbf{UninaDelivery}.

\section{Definizione dei Domini Personalizzati}

Vista la vasta gamma di vincoli di dominio simili su vari campi, si è deciso di definire dei domini personalizzati per facilitare la creazione delle tabelle.

\begin{lstlisting}
  CREATE DOMAIN letterString AS text CHECK (VALUE ~ '^[a-zA-Z]+$');
  
  CREATE DOMAIN numericString AS text CHECK (VALUE ~ '^[0-9]+$');
  
  CREATE DOMAIN alphanumericString AS text CHECK (VALUE ~ '^\w+[\w\s.]$');
  
  CREATE DOMAIN emailString AS text CHECK (
    VALUE ~ '^\w+[\w.]*\w@[a-zA-Z.]+\.[a-zA-Z]{2,}$'
  );
\end{lstlisting}

Inoltre per i campi che nel diagramma ristrutturato sono stati definiti come \textbf{enumerazioni}, si è deciso di utilizzare dei tipi enumerati personalizzati.
% TODO @zGenny @RiccardoElena we need to stlye this part
\begin{lstlisting}
  CREATE TYPE WorldZone AS ENUM (
    'NA', 'EUW', 'EUNE', 'LATAM', 'MIDEAST', 'CKJ', 'SEA', 'IND', 'RUS', 'STAN', 
    'OC', 'AFN', 'AFC', 'AFS'
  );

  CREATE TYPE DepositType AS ENUM (
    'City', 'State', 'Country', 'Central'
  );

  CREATE TYPE TransportType AS ENUM (
    'WheeledSmall', 'WheeledLarge', 'Rails', 'Water', 'Air'
  );
  
  CREATE TYPE DrivingLicenceType AS ENUM ('BE', 'CE');
\end{lstlisting}

\section{Creazione delle Tabelle}

Vediamo ora la creazione delle tabelle, con annessi vincoli di dominio e di n-upla, oltre che le chiavi primarie e le chiavi esterne.

\begin{lstlisting}[caption={Creazione della tabella \textbf{Area}}]
  CREATE TABLE Area (
    ZipCode numericString NOT NULL,
    City letterString NOT NULL,
    State letterString NOT NULL,
    Country letterString NOT NULL,
    WorldZone WorldZone NOT NULL
  );
  ALTER TABLE Area ADD CONSTRAINT Area_pk PRIMARY KEY (ZipCode, Country);
\end{lstlisting}

\begin{lstlisting}[caption={Creazione della tabella \textbf{Account}}]
  CREATE TABLE Account (
    Name letterString NOT NULL,
    Surname letterString NOT NULL,
    Email emailString NOT NULL,
    Birthdate date NOT NULL,
    ProPic text,
    Password text NOT NULL,
    AddressNo alphNumString NOT NULL,
    Street alphNumString NOT NULL,
    ZipCode numericString NOT NULL,
    Country letterString NOT NULL
  );
  ALTER TABLE account ADD CONSTRAINT Account_pk PRIMARY KEY (Email);
  ALTER TABLE account ADD CONSTRAINT validAccountBirthdate CHECK (
    EXTRACT( YEAR FROM age(Birthdate)) >= 16 
  );
  ALTER TABLE account ADD CONSTRAINT formatAccountPassword CHECK(
    Password ~ '^[A-Fa-f0-9]{64}$'
  );
  ALTER TABLE account ADD CONSTRAINT formatAccountProPic CHECK (
    ProPic ~ '^\/9j\/4AAQSkZJRg[-A-Za-z0-9+\/]*={0,2}$'
  );
  ALTER TABLE Account ADD CONSTRAINT formatAccountAddressNo CHECK (
    AddressNo ~ '^[1-9]\d*[a-z]?(?:BIS)?$'
  );
  ALTER TABLE Account ADD CONSTRAINT account_fk FOREIGN KEY (ZipCode, Country) REFERENCES Area(ZipCode, Country) ON DELETE CASCADE ON UPDATE CASCADE;
\end{lstlisting}
   % TODO define a trigger to avoid manual insertion of null email

\newpage
\begin{lstlisting}[caption={Creazione della tabella \textbf{Order}}]
  CREATE TABLE "Order"(
    OrderID SERIAL, 
    EmissionDate Date NOT NULL,
    IsExpress boolean,
    ExtraWarranty smallint NOT NULL,
    IsCompleted boolean,
    Email emailString
  );
  ALTER TABLE "Order" ADD CONSTRAINT Order_pk PRIMARY KEY (OrderID);
  ALTER TABLE "Order" ADD CONSTRAINT formatOrderExtraWarranty CHECK (
    ExtraWarranty >=0
  );
  ALTER TABLE "Order" ADD CONSTRAINT Order_fk FOREIGN KEY (Email) REFERENCES Account(Email) ON DELETE SET NULL ON UPDATE CASCADE;
  ALTER TABLE "Order" ADD CONSTRAINT formatOrderEmissionDate CHECK (
    EmissionDate <= current_date
  );
\end{lstlisting}

\begin{lstlisting}[caption={Creazione della tabella \textbf{Product}}]
  CREATE TABLE Product (
    Category letterString NOT NULL,
    Name text NOT NULL,
    Supplier alphNumString NOT NULL,
    Description text NOT NULL,
    PackageSizeLiters real NOT NULL,
    IsFragile boolean,
    Price numeric(100,2)
  );
  ALTER TABLE Product ADD CONSTRAINT Product_pk PRIMARY KEY (Name,Supplier);
  ALTER TABLE Product ADD CONSTRAINT formatProductName CHECK(
    Name ~ '^[a-zA-Z0-9]+[\ a-zA-Z0-9!@#$%^&*()_+{}\[\]:;<>,.?~\\\/-]*$'
  );
  ALTER TABLE Product ADD CONSTRAINT formatProductDescription CHECK(
    Description ~ '^[a-zA-Z0-9]+[\ a-zA-Z0-9!@#$%^&*()_+{}\[\]:;<>,.?~\\\/-]*$'
  );
  ALTER TABLE Product ADD CONSTRAINT formatProductPackageSizeLiters CHECK(
    PackageSizeLiters > 0
  );
  ALTER TABLE Product ADD CONSTRAINT formatProductPrice CHECK(
    Price > 0
  );
  ALTER TABLE Product ADD CONSTRAINT checkProductDescriptionOnTopic CHECK(
    (Description ILIKE ('% ' || Category || ' %')) AND (Description ILIKE ('% ' || Name || ' %')) AND (Description ILIKE ('% ' || Supplier || ' %'))
  );
\end{lstlisting}
% TODO define a trigger to avoid manual insertion of null name or supplier

\newpage
\begin{lstlisting}[caption={Creazione della tabella \textbf{Contains}}]
  CREATE TABLE Contains (
    OrderID integer NOT NULL,
    Name text ,
    Supplier alphNumString ,
    IsSent boolean,
    Quantity smallint NOT NULL
  );
  ALTER TABLE contains ADD CONSTRAINT noDuplicateProductPerOrder UNIQUE(OrderID, Name, Supplier);
  ALTER TABLE contains ADD CONSTRAINT formatContainsQuantity CHECK (Quantity > 0);
  ALTER TABLE contains ADD CONSTRAINT contains_fk_order FOREIGN KEY (OrderID) REFERENCES "Order"(OrderID) ON DELETE CASCADE ON UPDATE CASCADE;
  ALTER TABLE contains ADD CONSTRAINT contains_fk_product FOREIGN KEY (Name,Supplier) REFERENCES Product(Name, Supplier) ON DELETE RESTRICT ON UPDATE CASCADE;
\end{lstlisting}

\begin{lstlisting}[caption={Creazione della tabella \textbf{Deposit}}]
  CREATE TABLE Deposit (
    DepositID SERIAL NOT NULL,
    OccupiedSpace real NOT NULL,
    MaxCapacity real NOT NULL,
    DepositType DepositType,
    AddressNo alphNumString NOT NULL,
    Street alphNumString NOT NULL,
    ZipCode numericString NOT NULL,
    Country letterString NOT NULL
  );
  ALTER TABLE Deposit ADD CONSTRAINT Deposit_pk PRIMARY KEY ( DepositID);
  ALTER TABLE Deposit ADD CONSTRAINT foramtDepositOccupiedSpace CHECK(
    OccupiedSpace >= 0
  );
  ALTER TABLE Deposit ADD CONSTRAINT foramtDepositMaxCapacity CHECK(
    MaxCapacity > 0
  );
  ALTER TABLE Deposit ADD CONSTRAINT checkDepositFullness CHECK (
    OccupiedSpace <= MaxCapacity
  );
  ALTER TABLE Deposit ADD CONSTRAINT formatDepositAddressNo CHECK (
    AddressNo ~ '^[1-9]\d*[a-z]?(?:BIS)?$'
  );
  ALTER TABLE Deposit ADD CONSTRAINT Deposit_fk FOREIGN KEY (ZipCode, Country) REFERENCES Area(ZipCode, Country) ON DELETE CASCADE ON UPDATE CASCADE;
  ALTER TABLE deposit ADD CONSTRAINT onlyOneDepositPerAddress UNIQUE(AddressNo, Street, ZipCode, Country);
\end{lstlisting}

\newpage
\begin{lstlisting}[caption={Creazione della tabella \textbf{Transport}}]
  CREATE TABLE Transport (
    TransportID Serial NOT NULL,
    OccupiedSpace real NOT NULL,
    MaxCapacity real NOT NULL,
    IsAvailable boolean, 
    TransportType TransportType NOT NULL,
    DepositID integer NOT NULL
  );
  ALTER TABLE Transport ADD CONSTRAINT Transport_pk PRIMARY KEY(TransportID);
  ALTER TABLE Transport ADD CONSTRAINT Transport_fk FOREIGN KEY(DepositID) REFERENCES Deposit(DepositID) ON DELETE CASCADE ON UPDATE CASCADE; 
  ALTER TABLE Transport ADD CONSTRAINT formatTransportMaxCapacity CHECK (
    MaxCapacity >0
  );
  ALTER TABLE Transport ADD CONSTRAINT formatTransportOccupiedSpace CHECK (
    OccupiedSpace >=0
  );
  ALTER TABLE Transport ADD CONSTRAINT checkTransportFullness CHECK(
    OccupiedSpace <= MaxCapacity
  );
\end{lstlisting}

\begin{lstlisting}[caption={Creazione della tabella \textbf{Operator}}]
  CREATE TABLE Operator (
  Email emailString NOT NULL,
  BusinessMail emailString NOT NULL
);
ALTER TABLE Operator ADD CONSTRAINT Operator_pk PRIMARY KEY (BusinessMail);
ALTER TABLE Operator ADD CONSTRAINT Operator_fk FOREIGN KEY (Email) REFERENCES Account(Email) ON DELETE CASCADE ON UPDATE CASCADE;
\end{lstlisting}

\begin{lstlisting}[caption={Creazione della tabella \textbf{Covers}}]
  CREATE TABLE Covers (
    TransportID integer NOT NULL,
    ZipCode numericString NOT NULL,
    Country letterString NOT NULL,
    Date date
  );
  ALTER TABLE Covers ADD CONSTRAINT Covers_fk_Transport FOREIGN KEY (TransportID) REFERENCES Transport(TransportID) ON DELETE CASCADE ON UPDATE CASCADE;
  ALTER TABLE Covers ADD CONSTRAINT Covers_fk_Area FOREIGN KEY (ZipCode, Country) REFERENCES Area(ZipCode, Country) ON DELETE CASCADE ON UPDATE CASCADE;
  ALTER TABLE Covers ADD CONSTRAINT onlyOneAreaPerDay UNIQUE (TransportID, Date);
\end{lstlisting}

\newpage
\begin{lstlisting}[caption={Creazione della tabella \textbf{Shipment}}]
  CREATE TABLE Shipment (
    ShipmentID Serial NOT NULL,
    ShippingDate date NOT NULL,
    HasArrived boolean,
    ShippedFrom integer NOT NULL,
    DirectedTo integer,
    BusinessMail emailString NOT NULL,
    TransportID integer NOT NULL
  );
  ALTER TABLE Shipment ADD CONSTRAINT Shipment_pk PRIMARY KEY (ShipmentID);
  ALTER TABLE Shipment ADD CONSTRAINT Shipment_fk_startDeposit FOREIGN KEY (ShippedFrom) REFERENCES Deposit(DepositID) ON DELETE CASCADE ON UPDATE CASCADE;
  ALTER TABLE Shipment ADD CONSTRAINT Shipment_fk_arrivalDeposit FOREIGN KEY (DirectedTo) REFERENCES Deposit(DepositID) ON DELETE CASCADE ON UPDATE CASCADE;
  ALTER TABLE Shipment ADD CONSTRAINT checkDifferentStartEndDeposits CHECK (
    ShippedFrom <> DirectedTo
  );
  ALTER TABLE Shipment ADD CONSTRAINT validShipmentDate CHECK (
    ShippingDate >= current_date
  );
  ALTER TABLE Shipment ADD CONSTRAINT Shipment_fk_transport FOREIGN KEY (TransportID) REFERENCES Transport(TransportID) ON DELETE RESTRICT ON UPDATE CASCADE;
  ALTER TABLE Shipment ADD CONSTRAINT Shipment_fk_operator FOREIGN KEY (BusinessMail) REFERENCES Operator(Businessmail) ON DELETE CASCADE ON UPDATE CASCADE;
\end{lstlisting} % TODO make a trigger that block delete operator if exists unfinsihed shipment and set false to transport on his delete

\begin{lstlisting}[caption={Creazione della tabella \textbf{Ships}}]
  CREATE TABLE Ships ( 
    ShipmentID integer NOT NULL,
    OrderID integer NOT NULL
  );
  ALTER TABLE Ships ADD CONSTRAINT noDuplicateOrderInShipment UNIQUE (ShipmentID, OrderID);
  ALTER TABLE Ships ADD CONSTRAINT Ships_fk_Shipment FOREIGN KEY (ShipmentID) REFERENCES Shipment(ShipmentID) ON DELETE CASCADE ON UPDATE CASCADE;
  ALTER TABLE Ships ADD CONSTRAINT Ships_fk_Order FOREIGN KEY (OrderID) REFERENCES "Order"(OrderID) ON DELETE CASCADE ON UPDATE CASCADE;
\end{lstlisting}

\newpage
\begin{lstlisting}[caption={Creazione della tabella \textbf{Driver}}]
  CREATE TABLE Driver (
    BusinessMail emailString NOT NULL,
    DrivingLicenceType DrivingLicenceType NOT NULL,
    IsAvailable boolean,
    Email emailString, 
    DepositID integer NOT NULL 
  );
  ALTER TABLE Driver ADD CONSTRAINT Driver_pk PRIMARY KEY (BusinessMail);
  ALTER TABLE Driver ADD CONSTRAINT Driver_fk_Account FOREIGN KEY (Email) REFERENCES Account(Email) ON DELETE CASCADE ON UPDATE CASCADE;
  ALTER TABLE Driver ADD CONSTRAINT Driver_fk_Deposit FOREIGN KEY (DepositID) REFERENCES Deposit(DepositID) ON DELETE CASCADE ON UPDATE CASCADE;
\end{lstlisting}

\begin{lstlisting}[caption={Creazione della tabella \textbf{Drives}}]
  CREATE TABLE Drives(
    TransportID integer NOT NULL,
    BusinessMail emailString NOT NULL,
    Date date NOT NULL
  );
  ALTER TABLE Drives ADD CONSTRAINT Drives_fk_Driver FOREIGN KEY (BusinessMail) REFERENCES Driver(BusinessMail) ON DELETE CASCADE ON UPDATE CASCADE;
  ALTER TABLE Drives ADD CONSTRAINT Drives_fk_Transport FOREIGN KEY (TransportID) REFERENCES Transport(TransportID) ON DELETE CASCADE ON UPDATE CASCADE;
  ALTER TABLE Drives ADD CONSTRAINT onlyOneTravelPerDayTransport UNIQUE (TransportID, Date);
  ALTER TABLE Drives ADD CONSTRAINT onlyOneTravelPerDayDriver UNIQUE (BusinessMail, Date);
\end{lstlisting}

\begin{lstlisting}[caption={Creazione della tabella \textbf{Stores}}]
  CREATE TABLE Stores(
    Name text NOT NULL,
    Supplier text NOT NULL,
    DepositID integer NOT NULL,
    Quantity integer NOT NULL
  );
  ALTER TABLE Stores ADD CONSTRAINT noDuplicateProductInDeposit UNIQUE (Name, Supplier, DepositID);
  ALTER TABLE Stores ADD CONSTRAINT Stores_fk_Product FOREIGN KEY (Name, Supplier) REFERENCES Product(Name, Supplier) ON DELETE CASCADE ON UPDATE CASCADE;
  ALTER TABLE Stores ADD CONSTRAINT Stores_fk_Deposit FOREIGN KEY (DepositID) REFERENCES Deposit(DepositID) ON DELETE CASCADE ON UPDATE CASCADE;
  ALTER TABLE Stores ADD CONSTRAINT formatStoresQuantity CHECK(Quantity > 0);
\end{lstlisting}

\section{Creazione di Trigger e Funzioni}

Sono riportati alcuni dei trigger e delle funzioni creati per la gestione dei vincoli individuati e più in generale del database.

\subsection{Exception Handling}

Per una gestione più pulita degli errori si è deciso di creare una funzione che lanciasse tutte le eccezioni personalizzate, ricavandole da un'apposita tabella \textbf{custom\_error\_messages}.

Questa tabella si occupa di tener traccia dei codici di errore e dei messaggi di errore associati.

\begin{lstlisting}[caption={Creazione della tabella \textbf{custom\_error\_messages}}]
  CREATE TABLE custom_error_messages (
    error_code integer NOT NULL PRIMARY KEY,
    error_message text NOT NULL
  );
\end{lstlisting}

\begin{lstlisting}[caption={Creazione della funzione per gestire le eccezioni}]
  CREATE OR REPLACE FUNCTION raise_custom_error(p_error_code TEXT)
  RETURNS VOID AS $$
  DECLARE
    v_error_message TEXT;
  BEGIN
    -- Find error message in the custom_error_messages table 
    SELECT error_message INTO v_error_message
    FROM custom_error_messages
    WHERE error_code = p_error_code;
  
    -- Raise exception with the retrieved error message
    RAISE EXCEPTION USING 
              ERRCODE = p_error_code, 
              MESSAGE = v_error_message;
    
    -- If the error code given does not exist in the table, raise an exception to notify the user
    EXCEPTION
      WHEN NO_DATA_FOUND THEN
        RAISE EXCEPTION 'Error % not found', p_error_code;
  END;
  $$ LANGUAGE plpgsql;
\end{lstlisting}

Prima di ogni implementazione seguente, verranno riportati eventuali inserimenti nella tabella \textbf{custom\_error\_messages} per dare maggior contesto sulla gestione delle eccezioni interna ai frammenti di codice.

\subsection{Trigger per il vincolo \textbf{isAddressForSomethingSpecific}}

Per gestire questo vincolo sono necessari due trigger, uno per la tabella \textbf{Account} e uno per la tabella \textbf{Deposit}.

Nonostante ciò, la funzione che dovranno eseguire è praticamente analoga, fatto salvo per dei dettagli inerenti alla tabella su cui si sta lavorando.

Si avrà infatti in entrambi i casi che la funzione dovrà controllare nella tabella tra le due in cui non è chiamato se esistono righe con lo stesso \textit{AddressNo}, \textit{Street}, \textit{ZipCode} e \textit{Country}.

Dovrà inoltre tenere di conto del tipo di deposito. Infatti se si sta lavorando su \textbf{Account} si dovranno escludere dalla ricerca i \textbf{Deposit} di tipo \textit{City}, mentre se si sta lavorando su \textbf{Deposit} si dovrà controllare che il tipo di deposito inserito non sia \textit{City}.

Si può dunque utilizzare \textbf{SQL Dinamico} per generalizzare la funzione e utilizzarla in entrambi i trigger.

Le \textit{trigger function} però non possono accettare argomenti in ingresso. Per ovviare a ciò si è utilizzata la variabile \lstinline{TG_ARGV}, un array speciale messo a disposizione da PostgreSQL che contiene gli argomenti passati alla funzione all'interno di un trigger.

Iterando tra gli elementi di questo array si può ottenere il tipo di tabella su cui si sta lavorando e passarlo alla funzione.

\begin{lstlisting}[caption={Creazione errori personalizzati per \textbf{isAddressForSomethingSpecific}}]
  -- Insert custom error messages
  INSERT INTO custom_error_messages VALUES 
    ('E0002', 'This address is already occupied'),
    ('E0003', 'Incompatible table in input');
\end{lstlisting}

\begin{lstlisting}[caption={Creazione dei trigger per il vincolo \textbf{isAddressForSomethingSpecific}}]
  CREATE OR REPLACE TRIGGER isAddressForSomethingSpecificAccount
  BEFORE INSERT ON Account
  FOR EACH ROW
  EXECUTE PROCEDURE isAddressForSomethingSpecific_func('Deposit');

  CREATE OR REPLACE TRIGGER isAddressForSomethingSpecificDeposit
  BEFORE INSERT ON Deposit
  FOR EACH ROW
  EXECUTE PROCEDURE isAddressForSomethingSpecific_func('Account');
\end{lstlisting}

\newpage
\begin{lstlisting}[caption={Creazione della funzione \textbf{isAddressForSomethingSpecific\_func}}]
  CREATE OR REPLACE FUNCTION isAddressForSomethingSpecific_func()
  RETURNS TRIGGER
  AS 
  $$
    DECLARE 
      flag integer;
      queryp1 text = 'SELECT COUNT(*) 
                    FROM ';
      queryp2 text = ' 
                    WHERE (AddressNo = $1 AND 
                          Street = $2 AND 
                          ZipCode = $3 AND  
                          Country = $4';
      tableName text;
  BEGIN
      FOREACH tableName IN ARRAY TG_ARGV LOOP
  
        IF tableName = 'Deposit' THEN
          queryp2 = queryp2 || ' AND DepositType <> ''City'' ';
        ELSIF tableName = 'Account' THEN
          IF NEW.DepositType = 'City' THEN
            RETURN NEW;
          END IF;
        ELSE 
          perform raise_custom_error('E0003');
        END IF;
            
        queryp1 = queryp1 || tableName || queryp2 || ')';
        EXECUTE queryp1 into flag USING NEW.AddressNo, NEW.Street, NEW.ZipCode, NEW.Country;
        IF flag > 0 THEN
          perform raise_custom_error('E0002');
        END IF;
      END LOOP;
  
      RETURN NEW;
    END;
  $$ LANGUAGE plpgsql;
\end{lstlisting}

\subsection{Trigger per il vincolo \textbf{isShipmentDirectedInCorrectArea}}

Il vincolo è diviso in due casi, uno per le spedizioni dirette ai clienti e uno per le spedizioni dirette ai depositi.

In entrambi i casi però è necessario controllare che il trasporto associato alla spedizione abbia in programma per la data di spedizione la copertura della zona di destinazione.

In particolare il trasporto è adatto se non ha ancora zone programmate per quel giorno, oppure se ha come zona programmata proprio quella di destinazione.

In tal senso è possibile implementare questo vincolo utilizzando un corretto \textit{Exception Handling}.

Una volta recuperati i dati necessari a determinare la zona da coprire infatti possiamo dividere il problema in tre casi:

\begin{enumerate}
  \item Il trasporto non ha zone programmate per quel giorno: la spedizione è valida e va inserita la copertura per la zona della spedizione in \textbf{Covers}.
  \item Il trasporto ha come zona programmata proprio quella della spedizione: la spedizione è valida e non va inserita nessuna copertura.
  \item Il trasporto ha zone programmate per quel giorno ma non quella della spedizione: la spedizione non è valida e va lanciata un'eccezione.
\end{enumerate}

Possiamo notare ora che, fatta eccezione per il caso \(2\) risolvibile con un banale \lstinline{IF EXISTS}, gli altri due casi possono essere accorpati.

Infatti grazie al vincolo di unicità \textbf{onlyOneAreaPerDay} possiamo non preoccuparci di controllare se il trasporto è già impegnato in altra zona nella data di spedizione prima di inserire, poichè se ne occuperà \textbf{onlyOneAreaPerDay} lanciando un eccezione.

Trattando dunque separatamente il caso \(2\), che fortunamente è il più banale non dovendo fare nulla, possiamo accorpare i casi \(1\) e \(3\).

Per fare ciò andiamo a definire una funzione che si occupi di inserire la copertura per la zona di destinazione della spedizione, e che lanci un'eccezione se il trasporto è già impegnato in altra zona nella data di spedizione.

\begin{lstlisting}[caption={funzione per inserimento in \textbf{Covers}}]
  INSERT INTO custom_error_messages
      VALUES (
        'E0006',
        'Shipment assigned to a transport already occupied with diffrent location'
      );
  
  CREATE OR REPLACE FUNCTION createCoversForShipment(Zc Covers.ZipCode%TYPE, Cy Covers.Country%TYPE, Tid Covers.transportId%TYPE, D Covers.Date%TYPE)
  RETURNS VOID AS $$
  BEGIN
    -- Excluding Case 2 where nothig has to be done
    IF NOT EXISTS (
        SELECT 1
        FROM Covers
        WHERE ZipCode = ZC
          AND Country = Cy
          AND transportid = Tid
          AND Date = D
      ) THEN
        -- Trying to insert the new cover
        INSERT INTO Covers
          VALUES (Tid, Zc, Cy, D);
    END IF;
    -- If It's Case 1 the function will end here
  EXCEPTION
    WHEN unique_violation THEN 
      -- Otherwise if it's Case 3 an exception will be raised
      perform raise_custom_error('E0006');
  END;
  $$ LANGUAGE plpgsql;
\end{lstlisting}

È necessario adesso individuare il modo di raccogliere i dati necessari alla funzione nei due casi descritti sopra.

Nel caso in cui la spedizione sia diretta verso un deposito, i dati relativi alla zona da coprire sono accessibili direttamente con l'utilizzo della chiave esterna, mentre i dati realtivi a data e trasporto sono presenti nella riga da inserire.

Per quanto riguarda spedizioni dirette ai clienti invece, i dati relativi alla zona da coprire vanno recuperati nella tabella \textbf{Account}, passando per la tabella \textbf{Order} tramite la tabella \textbf{Ships}.

Avendo però \textbf{Ships} delle chiavi esterne verso \textbf{Order} e \textbf{Shipment}, non è possibile far attivare il trigger con un inserimento su \textbf{Shipment}, poiché questo stesso trigger dovrebbe andare a cercare i dati in \textbf{Ships} che non è ancora stata popolata.

Di conseguenza è necessario che il trigger si attivi su \textbf{Ships}, recuperando eventualmente i dati da \textbf{Shipment}.

Vediamo dunque prima i due trigger e successivamente le funzioni associate. 

La seconda delle due inoltre chiamerà una funzione che implementa il vincolo \textbf{isCityDepositShippingToClient}. 

Questo avviene in questa stessa \textit{trigger function} poiché l'evento di innesco è lo stesso, e quindi è possibile evitare di creare un ulteriore trigger.

Per questo vincolo è sufficiente controllare che, nel secondo caso di quelli trattati l'\textbf{Account} sia nella stessa \textbf{City} del \textbf{Deposit} di partenza della spedizione. 

\begin{lstlisting}[caption={Trigger per il vincolo \textbf{isShipmentDirectedInCorrectArea}}]
  CREATE OR REPLACE TRIGGER isShipmentDirectedInCorrectAreaDeposit 
  BEFORE INSERT ON Shipment 
  FOR EACH ROW 
  EXECUTE PROCEDURE isShipmentDirectedInCorrectAreaDeposit_func();

  CREATE OR REPLACE TRIGGER isShipmentDirectedInCorrectAreaAccount 
  BEFORE INSERT ON Ships
  FOR EACH ROW 
  EXECUTE PROCEDURE isShipmentDirectedInCorrectAreaAccount_func();
\end{lstlisting}

\begin{lstlisting}[caption={Funzione per le spedizioni verso depositi}]
  CREATE OR REPLACE FUNCTION isShipmentDirectedInCorrectAreaDeposit_func() 
  RETURNS TRIGGER AS $$
  DECLARE 
    ZCode Deposit.ZipCode%TYPE;
    Cy Deposit.Country%TYPE;
  BEGIN 
    -- Assuring it's the correct case
    IF NEW.DirectedTo IS NULL THEN 
      RETURN NEW;
    END IF;

    -- Retrieving the data needed
    SELECT ZipCode, Country 
    INTO ZCode, Cy
    FROM deposit
    WHERE depositid = NEW.DirectedTo;

    -- Calling the function to insert the cover
    perform createCoversForShipment(ZCode, Cy, NEW.transportId, NEW.ShippingDate);
    RETURN NEW;

  END;
  $$ LANGUAGE plpgsql;
\end{lstlisting}

\begin{lstlisting}[caption={Funzione per le spedizioni verso clienti}]
  CREATE OR REPLACE FUNCTION isShipmentDirectedInCorrectAreaAccount_func() 
  RETURNS TRIGGER AS $$
  DECLARE 
    ZCode Account.ZipCode%TYPE;
    Cy Account.Country%TYPE;
    Tid Shipment.transportId%TYPE;
    D Shipment.ShippingDate%TYPE;
  BEGIN 
    -- Assuring it's the correct case
    IF EXISTS (
      SELECT 1
      FROM Shipment
      WHERE ShipmentID=NEW.ShipmentID AND DirectedTo IS NOT NULL
    ) THEN
      RETURN NEW;
    END IF;
  
    -- Retrieving the data needed
    SELECT ZipCode, Country 
    INTO ZCode, Cy
    FROM "Order" NATURAL JOIN Account
    WHERE OrderId = NEW.OrderID;

    -- calling the function to check isCityDepositShippingToClient constraint
    perform isCityDepositShippingToClient(ZCode, Cy, NEW.shippedfrom);
  
    SELECT transportId, ShippingDate
    INTO Tid, D
    FROM shipment
    WHERE shipmentid = NEW.shipmentid;
  
    -- Calling the function to insert the cover
    perform createCoversForShipment(ZCode, Cy, Tid, D);
    RETURN NEW;
  END;
  $$ LANGUAGE plpgsql;
\end{lstlisting}

% TODO @zGenny last two constraints yet to be implemented. i'm trying to do them tomorrow