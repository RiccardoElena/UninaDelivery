\chapter{Schema Logico}

\section{Traduzione delle Associazioni}

\customTable{cY}[Dizionario delle classi Seconda Parte]{\textbfB{Associazione} & \textbfB{Strategia di Traduzione}}{
  \textbf{makes} & {\footnotesize Migrazione della \underline{chiave primaria} di \textbf{Account} in \textbf{Order}.} \\
  \textbf{contains} & {\footnotesize Definizione di una nuova tabella \textbf{Contains} con chiavi esterne rispettivamente verso \textbf{Order} e \textbf{Product}, oltre che gli attributi della relazione.} \\
  \textbf{stores} & {\footnotesize Definizione di una nuova tabella \textbf{Stores} con chiavi esterne verso \textbf{Product} e \textbf{Deposit}, oltre che gli attributi della relazione.} \\
  \textbf{worksAt} & {\footnotesize Migrazione della \underline{chiave primaria} di \textbf{Deposit} in \textbf{Driver}.} \\
  \textbf{belongs} & {\footnotesize Migrazione della \underline{chiave primaria} di \textbf{Deposit} in \textbf{Transport}.} \\
  \textbf{drives} & {\footnotesize Definizione di una nuova tabella \textbf{Drives} con chiavi esterne verso \textbf{Transport} e \textbf{Driver}, oltre che gli attributi della relazione.} \\
  \textbf{transportedBy} & {\footnotesize Migrazione della \underline{chiave primaria} di \textbf{Transport} in \textbf{Shipment}.} \\
  \textbf{covers} & {\footnotesize Definizione di una nuova tabella \textbf{Covers} con chiavi esterne verso \textbf{Transport} e \textbf{Area}, oltre che gli attributi della relazione.} \\
  \textbf{ships} & {\footnotesize Definizione di una nuova tabella \textbf{Ships} con chiavi esterne verso \textbf{Shipment} e \textbf{Order}.} \\
  \textbf{shippedFrom} & {\footnotesize Migrazione della \underline{chiave primaria} di \textbf{Deposit} in \textbf{Shipment}.} \\
  \textbf{shippedTo} & {\footnotesize Migrazione della \underline{chiave primaria} di \textbf{Deposit} in \textbf{Shipment}.} \\
  \textbf{is} & {\footnotesize Migrazione della \underline{chiave primaria} di \textbf{Account} in \textbf{Operator}.} \\
  \textbf{is} & {\footnotesize Migrazione della \underline{chiave primaria} di \textbf{Account} in \textbf{Driver}.} \\
  \textbf{inhabits} & {\footnotesize Migrazione della \underline{chiave primaria} di \textbf{Address} in \textbf{Account}.} \\
  \textbf{sited} & {\footnotesize Accorpamento delle classi coinvolte nella classe \textbf{Deposit}.}\\
  \textbf{located} & {\footnotesize Migrazione della \underline{chiave primaria} di \textbf{Address} in \textbf{Area}.} \\
}

\section{Traduzione delle Classi}

\subsection{Considerazioni sulla Traduzione delle Classi}

Dalla \textbf{traduzione delle associazioni} si è evidenziato come, per via delle strategie utilizzate per il mapping, la classe \textbf{Address} sia ridondante. Infatti la classe \textbf{Deposit} e la classe \textbf{Account} ne contengono tutti i campi, essendo tutti chiavi primarie, e per la classe \textbf{Area} la migrazione delle chiavi avveniva \textit{verso \textbf{Address}}, diventando anch'esse chiavi primarie e venendo ereditate da \textbf{Deposit} e \textbf{Account}.
È sufficiente quindi per eliminare la classe \textbf{Address} oltre alle strategie già messe in atto durante la \textbf{traduzione delle associazioni} migrare le chiavi primarie di \textbf{Area} direttamente verso \textbf{Deposit} e \textbf{Account}, andando a diminuire l'overhead e risolvendo in automatico vincoli come \intlink{isAddressForSomethingGeneral}{\textbf{isAddressForSomethingGeneral}}.

Per mantenere la consistenza del database, va anche introdotto un vincolo di unicità nella nuova classe accorpata \textbf{Deposit}. Per garantire l'associazione di tipo \(1:1\) tra \textbf{Deposit} e \textbf{Address} è necessario che la quadrupla di campi \textit{AddressNo}, \textit{Street}, \textit{ZipCode} e \textit{Country} sia unica nella tabella.
\section{Schema Logico}

\begin{note}[Indicazioni di Chiave Primaria e Chiave Esterna]
  Per evidenziare la presenza di \textbf{chiavi primarie} e \textbf{chiavi esterne} nei diagrammi, si è deciso sottolineare le \underline{chiavi primarie} e mettere in corsivo le \textit{chiavi esterne}.

  Per evitare superflue verbosità si è scelto di chiamare tutti i campi che rappresentano una chiave esterna con il nome della chiave primaria cui fanno riferimento. In caso di ambiguità, si è scelto di utilizzare il nome dell'associazione tradotta con l'uso di tale chiave o la classe a cui appartiene la chiave primaria a cui fa riferimento la chiave esterna.
\end{note}

% ? % TODO @zGenny need your opinion about the styling of this section. Is it good enough, or is it too empty?
% you can try your table idea here.

\textbf{Area} (\underline{ZipCode}, City, State, \underline{Country}, WorldZone) 

\textbf{Account} (Name, Surname, \underline{Email}, Birthdate, ProPic, Password, \textit{AddressNo}, \textit{Street}, \textit{ZipCode}, \textit{Country})

\textbf{Order} (\underline{OrderID}, EmissionDate, isExpress, ExtraWarranty, IsCompleted, \textit{Email})

\textbf{Contains} (\textit{OrderID}, \textit{Name}, \textit{Supplier}, IsSent, Quantity)

\textbf{Product} (Category, \underline{Name}, \underline{Supplier}, Description, PackageSizeLiters, isFragile, Price)

\textbf{Stores} (\textit{Name}, \textit{Supplier}, \textit{DepositID}, Quantity)

\textbf{Deposit} (\underline{DepositID}, OccupiedSpace, MaxCapacity, DepositType, \textit{AddressNo}, \textit{Street}, 

\textit{ZipCode}, \textit{Country})

\textbf{Shipment} (\underline{ShipmentID}, ShippingDate, HasArrived, \textit{ShippedFrom}, \textit{DirectedTo}, \textit{BusinessMail}, \textit{TransportID})

\textbf{Ships} (\textit{ShipmentID}, \textit{OrderID})

\textbf{Operator} (\underline{BusinessMail}, \textit{Email})

\textbf{Transport} (\underline{TransportID}, MaxCapacity, OccupiedSpace, IsAvailable, TransportType, \textit{DepositID}) 

\textbf{Drives} (\textit{TransportID}, \textit{BusinessMail}, Date)

\textbf{Driver} (\underline{BusinessMail}, DrivingLincenceType, IsAvailable, \textit{Email}, \textit{DepositID})

\textbf{Covers} (\textit{TransportID}, \textit{ZipCode}, \textit{Country}, Date)
