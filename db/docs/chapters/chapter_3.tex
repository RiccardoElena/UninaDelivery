\chapter{Schema Logico}

\section{Traduzione delle Associazioni}

\customTable{cY}[Dizionario delle classi Seconda Parte]{\textbfB{Associazione} & \textbfB{Strategia di Traduzione}}{
  \textbf{makes} & {\footnotesize Migrazione della \underline{chiave primaria} di \textbf{Account} in \textbf{Order}.} \\
  \textbf{contains} & {\footnotesize Migrazione della \underline{chiave primaria} di \textbf{Product} in \textbf{Order}.} \\
  \textbf{stores} & {\footnotesize Definizione di una nuova tabella \textbf{Stores} con chiavi esterne verso \textbf{Product} e \textbf{Deposit}, oltre che gli attributi della relazione.} \\
  \textbf{worksAt} & {\footnotesize Migrazione della \underline{chiave primaria} di \textbf{Deposit} in \textbf{Driver}.} \\
  \textbf{belongs} & {\footnotesize Migrazione della \underline{chiave primaria} di \textbf{Deposit} in \textbf{Transport}.} \\
  \textbf{drives} & {\footnotesize Definizione di una nuova tabella \textbf{Drives} con chiavi esterne verso \textbf{Transport} e \textbf{Driver}, oltre che gli attributi della relazione.} \\
  \textbf{transportedBy} & {\footnotesize Migrazione della \underline{chiave primaria} di \textbf{Transport} in \textbf{Shipment}.} \\
  \textbf{covers} & {\footnotesize Definizione di una nuova tabella \textbf{Covers} con chiavi esterne verso \textbf{Transport} e \textbf{Area}, oltre che gli attributi della relazione.} \\
  \textbf{ships} & {\footnotesize Definizione di una nuova tabella \textbf{Ships} con chiavi esterne verso \textbf{Shipment} e \textbf{Order}.} \\
  \textbf{shippedFrom} & {\footnotesize Migrazione della \underline{chiave primaria} di \textbf{Deposit} in \textbf{Shipment}.} \\
  \textbf{shippedTo} & {\footnotesize Migrazione della \underline{chiave primaria} di \textbf{Deposit} in \textbf{Shipment}.} \\
  \textbf{is} & {\footnotesize Migrazione della \underline{chiave primaria} di \textbf{Account} in \textbf{Operator}.} \\
  \textbf{is} & {\footnotesize Migrazione della \underline{chiave primaria} di \textbf{Account} in \textbf{Driver}.} \\
  \textbf{inhabits} & {\footnotesize Migrazione della \underline{chiave primaria} di \textbf{Address} in \textbf{Account}.} \\
  \textbf{sited} & {\footnotesize Accorpamento delle classi coinvolte nella classe \textbf{Deposit}.}\\
  \textbf{located} & {\footnotesize Migrazione della \underline{chiave primaria} di \textbf{Address} in \textbf{Area}.} \\
}

\section{Traduzione delle Classi}

\subsection{Considerazioni sulla Traduzione delle Classi}

Per quanto riguarda l'individuazione delle chiavi primarie, ciò è stato già fatto nella sezione \intlink{individuazioneDelleChiaviPrimarie}{identificazione delle chiavi primarie}.

Dalla \textbf{traduzione delle associazioni} si è evidenziato come, per via delle strategie utilizzate per la traduzione, la classe \textbf{Address} sia ridondante. Infatti la classe \textbf{Deposit} e la classe \textbf{Account} ne contengono tutti i campi, essendo tutti chiavi primarie, e per la classe \textbf{Area} la migrazione delle chiavi avveniva \textit{verso \textbf{Address}}, diventando anch'esse chiavi primarie e venendo ereditate da \textbf{Deposit} e \textbf{Account}.
È sufficiente quindi per eliminare la classe \textbf{Address} oltre alle strategie già messe in atto durante la \textbf{traduzione delle associazioni} migrare le chiavi primarie di \textbf{Area} direttamente verso \textbf{Deposit} e \textbf{Account}, andando a diminuire l'overhead e risolvendo in automatico vincoli come \intlink{isAddressForSomethingGeneral}{\textbf{isAddressForSomethingGeneral}}.

Per mantenere la consistenza del database, va anche introdotto un vincolo di unicità nella nuova classe accorpata \textbf{Deposit}. Per garantire l'associazione di tipo \(1:1\) tra \textbf{Deposit} e \textbf{Address} è necessario che la quadrupla di campi \textit{AddressNo}, \textit{Street}, \textit{ZipCode} e \textit{Country} sia unica nella tabella.
\section{Schema Logico}

\begin{note}[Indicazioni di Chiave Primaria e Chiave Esterna]
  Per evidenziare la presenza di \textbf{chiavi primarie} e \textbf{chiavi esterne} nei diagrammi, si è deciso sottolineare le \underline{chiavi primarie} e mettere in corsivo le \textit{chiavi esterne}.

  Per evitare superflue verbosità si è scelto di chiamare tutti i campi che rappresentano una chiave esterna con il nome della chiave primaria cui fanno riferimento. In caso di ambiguità, si è scelto di utilizzare il nome dell'associazione tradotta con l'uso di tale chiave o la classe a cui appartiene la chiave primaria a cui fa riferimento la chiave esterna.
\end{note}


\customTableLogic{|Y|Y|Y|Y|Y|}{5}{Area}{
  \underline{ZipCode} & City & State & \underline{Country} & WorldZone
}

\customTableLogic{|Y|Y|Y|Y|Y|}{5}{Account}{
  Name & Surname & \underline{Email} & Birthdate & ProPic \\
  \hline
  Password & \textit{AddressNo} & \textit{Street} & \textit{ZipCode} & \textit{Country}
}
 
\customTableLogic{|Y|Y|Y|Y|Y|}{5}{Order}{
  \underline{OrderID} & EmissionDate & isExpress & ExtraWarranty & IsCompleted \\
  \hline
  \textit{Email} & \textit{Quantity} & \textit{Name} & \textit{Supplier} & ---
}

\customTableLogic{|Y|Y|Y|Y|}{4}{Product}{
  Category & \underline{Name} & \underline{Supplier} & Description \\
  \hline
  PackageSizeLiters & isFragile & Price & ---
}

\customTableLogic{|Y|Y|Y|Y|}{4}{Stores}{
  \textit{Name} & \textit{Supplier} & \textit{DepositID} & Quantity
}

\customTableLogic{|Y|Y|Y|Y|}{4}{Deposit}{
  \underline{DepositID} & OccupiedSpace & MaxCapacity & DepositType \\
  \hline
  \textit{AddressNo} & \textit{Street} & \textit{ZipCode} & \textit{Country}
}

\customTableLogic{|Y|Y|Y|Y|}{4}{Shipment}{
  \underline{ShipmentID} & ShippingDate & HasArrived & \textit{ShippedFrom} \\
  \hline
  \textit{DirectedTo} & \textit{BusinessMail} & \textit{TransportID} & ---
}

\customTableLogic{|Y|Y|}{2}{Ships}{
  \textit{ShipmentID} & \textit{OrderID}
}

\customTableLogic{|Y|Y|}{2}{Operator}{
  \underline{BusinessMail} & \textit{Email}
}

\customTableLogic{|Y|Y|Y|}{3}{Transport}{
  \underline{TransportID} & MaxCapacity & IsAvailable \\
  \hline TransportType & \textit{DepositID} & ---
}

\customTableLogic{|Y|Y|Y|}{3}{Drives}{
  \textit{TransportID} & \textit{BusinessMail} & Date
}

\customTableLogic{|Y|Y|Y|Y|}{4}{Driver}{
  \underline{BusinessMail} & DrivingLincenceType & \textit{Email} & \textit{DepositID}
}

\customTableLogic{|Y|Y|Y|Y|Y|}{5}{Covers}{
  \textit{TransportID} & \textit{ZipCode} & \textit{Country} & Date & OccupiedSpace
}
