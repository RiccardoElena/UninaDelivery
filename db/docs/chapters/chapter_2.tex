\chapter{Progettazione Concettuale}

\section{Premesse alla lettura dei diagrammi}

\subsection{Modelli Utilizzati}

Procediamo alla modellizzazione del mini-mondo, partendo dalla progettazione concettuale.

Questa fase di progettazione è stata svolta utilizzando, oltre che il modello \textbf{UML} nella forma di un \textbf{Class Diagram}, anche il modello \textbf{EER}, ovvero \textbf{Enhanced Entity Relationship}, per cogliere meglio aspetti del dominio che un modello \textbf{ER} classico non avrebbe potuto cogliere, come ad esempio \textbf{generalizzazioni e specializzazioni}.

\subsection{Precisazioni sui Diagrammi}

\subsubsection{EER Diagram}
% TODO @zGenny you should update EER too.
% TODO @zGenny isCompleted field should not be derived anymore. To be updated EER, UML and UML Restructured
% TODO @zGenny remove isavailable field from operator too, it's useless. Do this in chapter 3 too.
% BUG : @RiccardoElena isAvaiable field from driver right? (Removed) Yes it was from driver, but it's not removed yet or am I missing something? 
Vista la densità del diagramma \textbf{EER}, si è deciso d'introdurre un \textbf{color coding} per facilitarne la lettura:

\begin{itemize}
  \item \textcolor{PRIMARY}{Entità}, e dunque Specializzazioni in \textcolor{PRIMARY}{arancione};
  \item \textcolor{CONTRAST}{Relazioni} in \textcolor{CONTRAST}{celeste};
  \item \textcolor{NEWGREEN}{Attributi} in \textcolor{NEWGREEN}{verde};
\end{itemize}

Inoltre, in caso di accavallamento di linee, si è deciso d'interrompere la linea in secondo piano in corrispondenza di un intersezione, così da evidenziare i diversi collegamenti.

\subsubsection{UML Class Diagram}

Per migliorare la leggibilità dei diagrammi, si è deciso di specificare le \textbf{molteplicità} degli attributi esclusivamente per sottolineare la possibilità di essere valorizzato a \textbf{NULL}. 

In tali casi si è utilizzata la molteplicità \textbf{[\(0..1\)]} esclusivamente per gli attributi di tipo \textbf{Bool}, mentre \textbf{[\(0..*\)]} per gli altri.

\bigskip

\begin{note}[Leggibilità dei Diagrammi]
  In caso di problemi di leggibilità dei diagrammi, sono disponibili le versioni originali nella pagina GitHub del progetto:
  \exlink{https://github.com/RiccardoElena/UninaDelivery/blob/develop/db/docs/sources/ER_Diagram.pdf}{ER Diagram} e \exlink{https://github.com/RiccardoElena/UninaDelivery/blob/develop/db/docs/sources/UML_Class_Diagram.pdf}{UML Class Diagram}.
\end{note}

\newpage

\section{Enhanced Entity Relationship Diagram}
\begin{center}
  \includegraphics[width=0.9\textwidth]{ER_Diagram.pdf}
\end{center}

\section{UML Class Diagram}
\begin{center}
  \includegraphics[width=\textwidth]{UML_Class_Diagram.pdf}
\end{center}

\newpage

\section{Ristrutturazione del Diagramma UML}

\subsection{Considerazioni sulla Ristrutturazione}

\subsubsection{Attributi Multipli e Multivalore}

Nello schema concettuale non sono presenti attributi \textbf{Multipli} o \textbf{Multivalore}, in quanto non sono stati ritenuti necessari per la rappresentazione del mini-mondo.

\subsubsection{Attributi Derivati}

Nello schema concettuale sono presenti due attributi \textbf{Derivati}:

\begin{itemize}
  \item L'attributo \textbf{Price} dell'entità \textbf{Shipment}, il quale però non ha necessità di essere conservato, non essendo un attributo di frequente richiesta per il sistema richiesto, e che quindi può essere eventualmente calcolato \textit{on-the-fly} in fase d'interrogazione del database basandosi sugli indirizzi di partenza e arrivo della spedizione e sulle eventuali modalità extra di consegna specificate in \textbf{Order}, come \textit{ExtraWarranty} e \textit{IsExpress}.
  \item L'attributo \textbf{OccupiedSpace} sia dell'entità \textbf{Deposit} che della classe d'associazione di \textbf{Covers}, infine, è stato anch'esso conservato, in quanto è un attributo che può essere richiesto frequentemente dal sistema, e richiede un interrogazione con risultato potenzialmente molto grande per essere valorizzata. 
\end{itemize}

\subsubsection{Generalizzazioni e Specializzazioni}

Le varie \textbf{Generalizzazioni} e \textbf{Specializzazioni} presenti nello schema concettuale sono state ristrutturate usando metodologie diverse, in base alla loro natura.

\begin{itemize}
  \item[\textbf{Employee:}] Essendo la specializzazione \textbf{Totale Disgiunta}, ed essendo le classi coinvolte in poche associazioni, si è scelto di accorpare la classe generale in quelle specializzate.
  \item[\textbf{Account:}] Essendo la specializzazione \textbf{Parziale Overlapping}, si è deciso di trasformarla in \textbf{Associazione}, limitando così il più possibile il numero di campi \textbf{NULL} e di \textbf{Vincoli d'Integrità}.
  \item[\textbf{Deposit:}] In questo caso trattandosi di una gerarchia di specializzazione si è scelto di accorpare a cascata le classi specializzate in quella generale, non avendo le classi specializzate attributi ed essendo tutte coinvolte in un'unica associazione concettualmente identica che verrà successivamente analizzata.
  \item[\textbf{Transport:}] Analogamente a quanto visto per \textbf{Deposit}, nonostante non si tratti di una gerarchia, si è scelto di accorpare le classi specializzate in quella generale. 
\end{itemize}

\subsubsection{Analisi delle Ridondanze}\label{Redundancy analysis}

Vediamo infine le ridondanze presenti nello schema concettuale, e come sono state rimosse.

Dalla ristrutturazione delle \textbf{Specializzazioni} di \textbf{Transport} sono emersi quattro attributi che servono essenzialmente lo stesso scopo ma per tipologie diverse di mezzi di trasporto: \textbf{Licence Plate}, \textbf{TrainNo}, \textbf{IMOCode} e \textbf{FlightNo}. Si è quindi deciso di accorpare questi attributi in un unico attributo \textbf{TransportID}, che può essere valorizzato con un identificativo univoco condiviso tra le tipologie di mezzi di trasporto, eliminando la possibilità di valori \textbf{NULL} o di valori uguali per tipi di mezzi diversi che imponevano aggiunta di vincoli o presenza di più attributi.

Inoltre la ristrutturazione delle \textbf{Specializzazioni} di \textbf{Deposit} ha fatto emergere quattro associazioni identiche a meno di vincoli che le riguardano, ovvero le associazioni \textbf{belongs}. È possibile dunque ridurle a un'unica associazione preservando i vincoli.

In ultima analisi, la classe \textbf{Shipment} è coinvolta in due associazioni semanticamente identiche, che si distinguono solo per molteplicità e seconda classe coinvolta, ovvero \textbf{Account} e \textbf{Deposit}. In particolare quella con \textbf{Account} è un'associazione circolare, essendo entrambe le classi associate anche con \textbf{Order}. Si è quindi deciso di eliminare l'associazione con \textbf{Account}, essendo il destinatario della spedizione già noto tramite \textbf{Order}.

\subsection{Identificazione delle Chiavi primarie}\label{individuazioneDelleChiaviPrimarie}

Aldilà delle chiavi primarie già presenti nello schema concettuale esplicitate nel \textbf{diagramma ER} e dell'attributo \textbf{TransportID} introdotto \intlink{Redundancy analysis}{precedentemente}, si è deciso di aggiungere chiavi surrogate per le entità \textbf{Order}, \textbf{Shipment} e \textbf{Deposit}, poiché migrando le chiavi di altre entità si sarebbero ottenute chiavi composte troppo lunghe e complesse.

Per l'entità \textbf{Address} invece, è risultato più appropriato migrare la chiave primaria di \textbf{Area} essendo una composizione e di conseguenza già concettualmente una \textit{parte} dell'entità \textbf{Address}. 

\newpage

\subsection{Class Diagram Ristrutturato}

\begin{center}
  \includegraphics[width=\textwidth]{UML_Class_Diagram_Restructured.pdf} 
\end{center}

\newpage

\section{Dizionari}

\subsection{Dizionario delle Classi}

\customTable{cYY}[Dizionario delle classi Prima Parte]{\textbfB{Classe} & \textbfB{Descrizione} & \textbfB{Attributi}}{
  \textbf{Account} & Generico utente che utilizza il sistema per effettuare ordini. & 
  {\footnotesize 
  \textbf{Name:} (\textit{string}): Nome dell'utente;
  
  \textbf{Surname} (\textit{string}): Cognome dell'utente;
  
  \textbf{Email} (\textit{string}): Chiave Primaria. Email dell'utente;

  \textbf{Birthdate} (\textit{date}): Data di nascita dell'utente;

  \textbf{ProPic} (\textit{string}): Eventuale immagine profilo dell'utente;

  \textbf{Password} (\textit{string}): Password dell'utente;  
  }\\


  \textbf{Operator} & Account di un impiegato che si occupa di creare le spedizioni. &
  {\footnotesize
  \textbf{BusinessMail} (\textit{string}): Chiave Primaria. Email aziendale dell'operatore;
  }\\


  \textbf{Driver} & Account di un autista che si occupa di effettuare le consegne. &
  {\footnotesize
  \textbf{BusinessMail} (\textit{string}): Chiave Primaria. Email aziendale dell'autista;

  \textbf{DrivingLincenceType} (\textit{DrivingLincenceType}): Tipologia di patente dell'autista;
  }\\


  \textbf{Order} & Ordine effettuato da un utente. Può contenere più prodotti e può essere spedito in più spedizioni. &
  {\footnotesize
  \textbf{OrderID} (\textit{integer}): Chiave Surrogata. Identificativo dell'ordine;

  \textbf{EmissionDate} (\textit{date}): Data di emissione dell'ordine;

  \textbf{IsExpress} (\textit{bool}): Indica se l'ordine è espresso o meno;

  \textbf{ExtraWarranty} (\textit{bool}): Indica se è stata acquistata una garanzia extra o meno;

  \textbf{IsCompleted} (\textit{bool}): Indica se l'ordine è stato completato o meno;
  }\\


  \textbf{Shipment} & Singola spedizione partita da un deposito. Può contenere più ordini e può raggiungere a più utenti o a un deposito. &
  {\footnotesize
  \textbf{ShipmentID} (\textit{integer}): Chiave Surrogata. Identificativo della spedizione; 

  \textbf{ShippingDate} (\textit{date}): Data della spedizione;
  }\\  
}

\newpage

\customTable{cYY}[Dizionario delle classi Seconda Parte]{\textbfB{Classe} & \textbfB{Descrizione} & \textbfB{Attributi}}{
  \textbf{Product} & Prodotto acquistabile da un utente. Può essere conservato in più depositi. &
  { \footnotesize
    \textbf{Type} (\textit{string}): Categoria del prodotto;

    \textbf{Name} (\textit{string}): Chiave Primaria. Nome del prodotto;

    \textbf{Supplier} (\textit{string}): Chiave Primaria. Fornitore del prodotto. Di default è \textit{UninaDelivery};
  
    \textbf{Description} (\textit{string}): Descrizione del prodotto;

    \textbf{PackageSizeLiters} (\textit{float}): Dimensione del prodotto in litri;

    \textbf{IsFagile} (\textit{bool}): Indica se il prodotto è fragile o meno;

    \textbf{Price} (\textit{float}): Prezzo del prodotto;
  }\\


  \textbf{Deposit} & Deposito per la conservazione di prodotti. Può contenere più prodotti e può essere raggiunto da più spedizioni. &
  {\footnotesize
  \textbf{DepositID} (\textit{integer}): Chiave Surrogata. Identificativo del deposito;

  \textbf{OccupiedSpace} (\textit{float}): Spazio del deposito attualmente occupato;

  \textbf{MaxCapacity} (\textit{float}): Spazio massimo del deposito;

  \textbf{DepositType} (\textit{DepositType}): Tipologia del deposito;
  }\\
  

  \textbf{Area} & Zona del mondo raggiungibile da un mezzo di trasporto. Contiene più indirizzi. &
  {\footnotesize
  \textbf{ZipCode} (\textit{string}): Chiave Primaria. Codice di Avviamento Postale della zona;

  \textbf{City} (\textit{string}): Città in cui si trova la zona;

  \textbf{Country} (\textit{string}): Chiave Primaria. Paese in cui si trova la zona;
  
  \textbf{WorldZone} (\textit{string}): Zona del mondo in cui si trova la zona;
  }\\

  \textbf{Address} & Indirizzo di un utente o di un deposito. & 
  {\footnotesize

  \textbf{AddressNo} (\textit{string}): Chiave Primaria. Numero civico dell'indirizzo; 

  \textbf{Street} (\textit{string}): Via dell'indirizzo;
  }\\

  \textbf{Transport} & Mezzo di trasporto utilizzato per le spedizioni. & 
  {\footnotesize
  \textbf{TransportID} (\textit{integer}): Chiave Surrogata. Identificativo del mezzo di trasporto;

  \textbf{MaxCapacity} (\textit{float}): Capacità massima del mezzo di trasporto;

  \textbf{IsAvailable} (\textit{bool}): Indica se il mezzo di trasporto è disponibile o meno. Il campo sarà valorizzato a \textit{true} solo se il mezzo non è già in viaggio;
  
  \textbf{TransportType} (\textit{TransportType}): Tipologia del mezzo di trasporto;
  }\\

}

\subsection{Dizionario delle Associazioni}

\customTable{cYYY}[Dizionario delle associazioni Prima Parte]{\textbfB{Associazione} & \textbfB{Descrizione} & \textbfB{Attributi} & \textbfB{Classi Coinvolte}}{
  \textbf{Makes} & 
  {\footnotesize
  Esprime la relazione tra un \textbf{Account} e un \textbf{Order}. Un \textbf{Account} può effettuare più \textbf{Order}, mentre un \textbf{Order} è effettuato da un solo \textbf{Account}.
  } & 
  --- & 
  {\footnotesize
  \textbf{Account \([1]\)} (\textit{makes}): L'account che effettua un dato ordine;

  \textbf{Order \([*]\)} (\textit{made by}): Gli ordini effettuati da un dato account;
  
  }\\

  \textbf{Inhabits} &
  {\footnotesize
  Esprime la relazione tra un \textbf{Account} e un \textbf{Address}. Un \textbf{Account} può avere un solo \textbf{Address}, mentre un \textbf{Address} può corrispondere a più \textbf{Account}.
  } &
  --- &
  {\footnotesize
  \textbf{Account \([0..*]\)} (\textit{inhabits}): Gli account che abitano in un dato indirizzo;

  \textbf{Address \([1]\)} (\textit{is inhabited by}): L'indirizzo abitato da un dato account;
  }\\

  \textbf{Sited} &
  {\footnotesize
  Esprime la relazione tra un \textbf{Address} e un \textbf{Deposito}. Un \textbf{Address} può corrispondere a un solo \textbf{Deposit}, mentre un \textbf{Deposit} ha solo un \textbf{Address}.
  } &
  --- &
  {\footnotesize
  \textbf{Address \([1]\)} (\textit{is site of}): L'indirizzo di un dato deposito;

  \textbf{Deposit \([1]\)} (\textit{sited on}): Il deposito che si trova in un dato indirizzo;
  
  }\\
  
  \textbf{Located} &
  {\footnotesize
  Esprime la relazione tra \textbf{Area} e \textbf{Address}. Un \textbf{Area} può contenere più \textbf{Address}, e un \textbf{Address} appartiene a un'unica \textbf{Area}.
  } &
  --- &
  {\footnotesize
  \textbf{Area \([*]\)} (\textit{is location}): Le Area che contengono un dato indirizzo;

  \textbf{Address \([*]\)} (\textit{located}): Gli indirizzi che appartengono a una data zona;
  }\\

  \textbf{Contains} &
  {\footnotesize
  Esprime la relazione tra \textbf{Order} e \textbf{Product}. Un \textbf{Order} può contenere una quantità N di \textbf{Product} e un \textbf{Product} può essere contenuto in più \textbf{Order}. 
  } &
  --- 
  &
  {\footnotesize
  \textbf{Order \([*]\)} (\textit{contains}): Gli ordini che contengono un dato prodotto;

  \textbf{Product \([1]\)} (\textit{is contained}): Il prodotto contenuto in un dato ordine;
  }\\
}
\customTable{cYYY}[Dizionario delle associazioni Seconda Parte]{\textbfB{Associazione} & \textbfB{Descrizione} & \textbfB{Attributi} & \textbfB{Classi Coinvolte}}{
  \textbf{Stores} &
  {\footnotesize
  Esprime la relazione tra \textbf{Deposit} e \textbf{Product}. Un \textbf{Deposit} può contenere più \textbf{Product}, mentre un \textbf{Product} può essere contenuto in più \textbf{Deposit}.
  } &
  {\footnotesize

  \textbf{Quantity} (\textit{integer}): Numero di pezzi di un dato prodotto conservati in un deposito; 
  } &
  
  {\footnotesize
  \textbf{Deposit \([*]\)} (\textit{stores}): I depositi che contengono un dato prodotto;

  \textbf{Product \([*]\)} (\textit{is stored in}): I prodotti contenuti in un dato deposito;
  
  }\\

  \textbf{ShippedFrom} &
  {\footnotesize
  Esprime la relazione tra \textbf{Shipment} e \textbf{Deposit} \textit{mittente}. Una \textbf{Shipment} può avere un solo mittente, mentre un \textbf{Deposit} può essere mittente di più \textbf{Shipment}.
  } &
  --- &
  {\footnotesize
  \textbf{Shipment \([*]\)} (\textit{shipped}): Le spedizioni che partono da un dato deposito; 

  \textbf{Deposit \([1]\)} (\textit{seder}): Il deposito mittente di una data spedizione; 
  }\\
  
  \textbf{Ships} &
  {\footnotesize
  Esprime la relazione tra \textbf{Shipment} e \textbf{Order}. Una \textbf{Shipment} può contenere più \textbf{Order}, mentre un \textbf{Order} può essere contenuto in più \textbf{Shipment}.
  } &
  --- &
  {\footnotesize
  \textbf{Shipment \([*]\)} (\textit{ships}): Le spedizioni che spediscono un dato ordine; 

  \textbf{Order \([*]\)} (\textit{is shipped}): Gli ordini spediti da una data spedizione; 
  }\\

  \textbf{Creates} &
  {\footnotesize
  Esprime la relazione tra l'\textbf{Operator} e l'\textbf{Order}. Un \textbf{Operator} può creare più \textbf{Order}, mentre un \textbf{Order} può essere creato da un solo \textbf{Operator}.
  } &
  --- &
  {\footnotesize
  \textbf{Operator \([1]\)} (\textit{creates}): L'operatore che crea una data spedizione; 

  \textbf{Shipment \([*]\)} (\textit{is created}): Le spedizioni create da un dato operatore; 
  }\\

  \textbf{TransportedBy} &
  {\footnotesize
  Esprime la relazione tra \textbf{Shipment} e \textbf{Transport}. Una \textbf{Shipment} può essere trasportata da un solo \textbf{Transport}, mentre un \textbf{Transport} può trasportare più \textbf{Shipment}.
  } &
  --- &
  {\footnotesize
  \textbf{Shipment \([*]\)} (\textit{transported by}): Le spedizioni trasportate da un dato mezzo di trasporto;; 

  \textbf{Transport \([1]\)} (\textit{transports}): Il mezzo di trasporto che trasporta una data spedizione; 
  }\\
  
}
\customTable{cYYY}[Dizionario delle associazioni Terza Parte]{\textbfB{Associazione} & \textbfB{Descrizione} & \textbfB{Attributi} & \textbfB{Classi Coinvolte}}{
  
  \textbf{Belongs} &
  {\footnotesize
  Esprime la relazione tra \textbf{Transport} e \textbf{Deposit}. Un \textbf{Transport} può appartenere a un solo \textbf{Deposit}, mentre un \textbf{Deposit} può avere più \textbf{Transport}.
  } &
  --- &
  {\footnotesize
  \textbf{Deposit \([1]\)} (\textit{is belonged}): Il deposito a cui appartiene un dato mezzo di trasporto; 

  \textbf{Transport \([*]\)} (\textit{belongs}): I mezzi di trasporto che appartengono a un dato deposito; 
  }\\

  \textbf{Is} &
  {\footnotesize
  Esprime la relazione tra \textbf{Account} e \textbf{Driver}. Indica se un \textbf{Account} oltre a essere un cliente è anche un \textbf{Driver}.
  } &
  --- &
  {\footnotesize
  \textbf{Account \([1]\)} (\textit{is}): I dati generali dell'account di un dato autista; 

  \textbf{Driver \([0..1]\)} (\textit{has}): I dati specifici dell'autista di un dato account;
  }\\

  \textbf{Is} &
  {\footnotesize
  Esprime la relazione tra \textbf{Account} e \textbf{Operator}. Indica se un \textbf{Account} oltre a essere un cliente è anche un \textbf{Operator}.
  } &
  --- &
  {\footnotesize
  \textbf{Account \([1]\)} (\textit{is}): I dati generali dell'account di un dato operatore; 

  \textbf{Operator \([0..1]\)} (\textit{has}): I dati specifici dell'operatore di un dato account;
  }\\

  \textbf{WorksAt} &
  {\footnotesize
  Esprime la relazione tra \textbf{Driver} e \textbf{Deposit}. Un \textbf{Driver} può lavorare in un solo \textbf{Deposit}, mentre un \textbf{Deposit} può avere più \textbf{Driver}.
  } &
  --- &
  {\footnotesize
  \textbf{Driver \([0..*]\)} (\textit{works at}): Gli autisti che lavorano per un dato deposito; 

  \textbf{Deposit \([1]\)} (\textit{employs}): Il deposito che impiega un dato autista; 
  }\\

  \textbf{Covers} &
  {\footnotesize
  Esprime la relazione tra \textbf{Transport} e \textbf{Area}. Un \textbf{Transport} può coprire più \textbf{Area}, mentre una \textbf{Area} può essere coperta da più \textbf{Transport}.
  } &
  {\footnotesize

  \textbf{Date} (\textit{date}): Data in cui un trasporto copre una certa zona. Un trasporto non può coprire più Area nella stessa data; 
  
  \textbf{OccupiedSpace} (\textit{float}): Spazio attualmente occupato dal mezzo di trasporto;

  } &
  {\footnotesize
  \textbf{Area \([*]\)} (\textit{is covered}): Le Area coperte da un dato mezzo di trasporto;

  \textbf{Transport \([*]\)} (\textit{covers}): I trasporti che coprono una data zona. Anche più di uno per data;
  }\\

  \textbf{Drives} &
  {\footnotesize
  Esprime la relazione tra \textbf{Driver} e \textbf{Transport}. Un \textbf{Driver} può guidare più \textbf{Transport}, mentre un \textbf{Transport} può essere guidato da più \textbf{Driver}.
  
  } &
  {\footnotesize

  \textbf{Date} (\textit{date}): Data in cui un autista giuda un mezzo di trasporto. Un trasporto non può essere guidato da più autisti nella stessa data e viceversa; 
  
  } &
  {\footnotesize
  \textbf{Driver \([*]\)} (\textit{drives}): Gli autisti che guidano per un dato mezzo di trasporto; 
  \textbf{Transport \([*]\)} (\textit{is driven}): I mezzi di trasporto guidati da un dato autista;
  }\\
}

\subsection{Dizionario dei Vincoli}\label{ConstraintDictionary}

\customTable{ccY}[Dizionario dei vincoli Prima Parte]{\textbfB{Vincolo} & \textbfB{Tipologia} & \textbfB{Descrizione}}{
  \textbf{formatAccountName} & Dominio &
  {\footnotesize
  Il campo \textbf{Name} della classe \textbf{Account} deve contenere solo caratteri alfabetici
  }\\

  \textbf{formatAccountSurname} & Dominio &
  {\footnotesize
  Il campo \textbf{Surname} della classe \textbf{Account} deve contenere solo caratteri alfabetici
  }\\

  \textbf{formatAccountEmail} & Dominio &
  {\footnotesize
  Il campo \textbf{Email} della classe \textbf{Account} deve avere il formato `\textbf{word}@\textbf{domain}.\textbf{ext}', dove \textit{word} può contenere solo caratteri alfanumerici o al più dei punti, \textit{domain} da soli caratteri alfabetici o punti, mentre \textit{ext} da solo caratteri alfabetici, e dev'essere lunga almeno due caratteri.
  }\\

  \textbf{validOperatorBusinessmail} & Dominio &
  {\footnotesize
  Il campo \textbf{Businessmail} della classe \textbf{Operator} deve avere il formato `\textbf{word}@\textbf{domain}.\textbf{ext}', dove \textit{word} può contenere solo caratteri alfanumerici o al più dei punti, \textit{domain} da soli caratteri alfabetici o punti, mentre \textit{ext} da solo caratteri alfabetici, e dev'essere lunga almeno due caratteri.
  }\\

  \textbf{validDriverBusinessmail} & Dominio &
  {\footnotesize
  Il campo \textbf{Businessmail} della classe \textbf{Driver} deve avere il formato `\textbf{word}@\textbf{domain}.\textbf{ext}', dove \textit{word} può contenere solo caratteri alfanumerici o al più dei punti, \textit{domain} da soli caratteri alfabetici o punti, mentre \textit{ext} da solo caratteri alfabetici, e dev'essere lunga almeno due caratteri.
  }\\
  
  \textbf{validAccountBirthdate} & Dominio & 
  {\footnotesize
  L'utente deve aver compiuto 16 anni
  }\\

  \textbf{formatAccountProPic} & Dominio &
  {\footnotesize
  Il campo \textbf{ProPic} deve essere una stringa in formato \textit{Base64} compatibile con un immagine in formato `JPEG'. 
  }\\

  \textbf{formatAccountPassword} & Dominio &
  {\footnotesize
  Il campo \textbf{Password} della classe \textbf{Account} dev'essere una valida stringa codificata dall'algoritmo HASH SHA-256.
  }\\

}

\customTable{ccY}[Dizionario dei vincoli Seconda Parte]{\textbfB{Vincolo} & \textbfB{Tipologia} & \textbfB{Descrizione}}{

  \textbf{formatTransportMaxCapacity} & Dominio &
  {\footnotesize
  Il campo \textbf{MaxCapacity} della classe \textbf{Transport} dev'essere positivo.
  }\\

  \textbf{formatCoversOccupiedSpace} & Dominio & 
  {\footnotesize
  Il campo \textbf{OccupiedSpace} della classe \textbf{Covers} dev'essere non negativo.
  }\\

  \textbf{formatOrderExtraWarranty} & Dominio &
  {\footnotesize
  Il campo \textbf{ExtraWarranty} della classe \textbf{Order} deve essere non negativo.
  }\\

  \textbf{formatOrderEmissionDate} & Dominio &
  {\footnotesize
    Il campo \textbf{EmissionDate} della classe \textbf{Order} non può essere successiva alla data corrente.
  }\\

  \textbf{formatAddressAddressNo} & Dominio &
  {\footnotesize
  Il campo \textbf{AddressNo} della classe \textbf{Address} dev'essere una stringa formata da sole cifre, seguita al più da una lettera, eventualmente seguita dalla parola `BIS'.
  }\\
  \textbf{formatAddressStreet} & Dominio &
  {\footnotesize
  Il campo \textbf{Street} della classe \textbf{Address} può contenere solo caratteri alfanumerici o al più il carattere \textit{`.'}. %chktex 38
  }\\
  
  \textbf{formatDepositOccupiedSpace} & Dominio & 
  {\footnotesize
  Il campo \textbf{OccupiedSpace} della classe \textbf{Deposit} deve essere un float non negativo.
  }\\
  
  \textbf{formatDepositMaxCapacity} & Dominio & 
  {\footnotesize
  Il campo \textbf{MaxCapacity} della classe \textbf{Deposit} deve essere un float positivo.
  }\\
  
  \textbf{formatProductCategory} & Dominio  & 
  {\footnotesize
  Il campo \textbf{Category} della classe \textbf{Product} deve contenere solo caratteri alfabetici.
  }\\
  
  \textbf{formatProductName} & Dominio  & 
  {\footnotesize
  Il campo \textbf{Name} della classe \textbf{Product} deve contenere caratteri alfanumerici, tra cui necessariamente il primo, e può contenere caratteri speciali. 
  }\\
  
  \textbf{formatProductSupplier} & Dominio  & 
  {\footnotesize
  Il campo \textbf{Supplier} della classe \textbf{Product} deve contenere caratteri alfanumerici.
  }\\
  
  \textbf{formatProductDescription} & Dominio & 
  {\footnotesize
  Il campo \textbf{Description} della classe \textbf{Product} deve contenere caratteri alfanumerici e può contenere caratteri speciali. 
  }\\
  
}

\customTable{ccY}[Dizionario dei vincoli Terza Parte]{\textbfB{Vincolo} & \textbfB{Tipologia} & \textbfB{Descrizione}}{
    
  \textbf{formatProductPackageSizeLiters} & Dominio & 
  {\footnotesize
  Il campo \textbf{PackageSizeLiters} della classe \textbf{Product} deve essere positivo.
  }\\ 

  \textbf{formatProductPrice} & Dominio & 
  {\footnotesize
  Il campo \textbf{Price} della classe \textbf{Product} deve essere positivo.
  }\\

  \textbf{formatStoresQuantity} & Dominio & 
  {\footnotesize
  Il campo \textbf{Quantity} della classe d'associazione \textbf{Stores} deve essere non negativo.
  }\\ 

  \textbf{formatOrderQuantity} & Dominio & 
  {\footnotesize
  Il campo \textbf{Quantity} della classe \textbf{Order} deve essere positivo.
  }\\

  \textbf{formatAreaZipCode} & Dominio & 
  {\footnotesize
  Il campo \textbf{ZipCode} della classe \textbf{Area} deve contenere solo caratteri numerici.
  }\\

  \textbf{formatAreaCity} & Dominio & 
  {\footnotesize
  Il campo \textbf{City} della classe \textbf{Area} deve contenere solo caratteri alfabetici.
  }\\
  
  \textbf{formatAreaState} & Dominio & 
  {\footnotesize
  Il campo \textbf{State} della classe \textbf{Area} deve contenere solo caratteri alfabetici.
  }\\
  
  \textbf{formatAreaCountry} & Dominio & 
  {\footnotesize
  Il campo \textbf{Country} della classe \textbf{Area} deve contenere solo caratteri alfabetici.
  }\\
  
  \textbf{checkDepositFullness} & N-upla & 
  {\footnotesize
  Nella classe \textbf{Deposit} il campo \textbf{OccupiedSpace} può avere valori solo minori o uguali al campo \textbf{MaxCapacity}
  }\\
  
  \textbf{checkTransportFullness} & Relazionale & 
  {\footnotesize
  Nella classe d'associazione \textbf{Cover} il campo \textbf{OccupiedSpace} può avere valori solo minori o uguali al campo \textbf{MaxCapacity} della classe \textbf{Transport}
  }\\
  
  \textbf{validShipmentDate} & Dominio & 
  {\footnotesize
  Il Campo \textbf{ShipmentDate} deve essere una data successiva alla data dell'inserimento.
  }\\

  \textbf{checkProductDescriptionOnTopic} & N-upla &
  {\footnotesize
  Nella classe \textbf{Product} il campo \textbf{Description} dev'essere una stringa che contenga le stringhe presenti nei campi \textbf{Category}, \textbf{Name} e \textbf{Supplier}
  }\\

  \textbf{checkDifferentStartEndDeposits} & N-upla &
  {\footnotesize
  Nella classe \textbf{Shipment} il campo \textbf{ShippedFrom} dev'essere diverso dal campo \textbf{DirectedTo}
  }\\
  
}

\customTable{ccY}[Dizionario dei vincoli Quarta Parte]{\textbfB{Vincolo} & \textbfB{Tipologia} & \textbfB{Descrizione}}{
  

  \textbf{onlyOneTravelPerDayTransport} & Intrarelazionale & 
  {\footnotesize
  Nella classe d’associazione \textbf{Drives} un \textbf{Transport} non può essere guidato più volte nella stessa \textbf{data}
  }\\

  \textbf{onlyOneTravelPerDayDriver} & Intrarelazionale &
  {\footnotesize
  Nella classe d’associazione \textbf{Drives} un \textbf{Driver} non può guidare più volte nella stessa \textbf{data}.
  }\\

  \textbf{onlyOneAreaPerDay} & Intrarelazionale & 
  {\footnotesize    
  Nella classe d’associazione \textbf{Covers} un \textbf{Transport} non può essere impiegato nella stessa data
  }\\

  \textbf{validEmployeeBirthdate} & Interrelazionale &
  {\footnotesize
  Se \textbf{Account} è un \textbf{Driver} o un \textbf{Operator} deve aver compiuto 18 anni
  }\\

  \textbf{isAddressForSomethingGeneral}\label{isAddressForSomethingGeneral} & Interrelazionale &
  {\footnotesize
  Un \textbf{Address}, non può non corrispondere né a un \textbf{Deposit}, né a degli \textbf{Account}.
  }\\

  \textbf{isAddressForSomethingSpecific} & Interrelazionale &
  {\footnotesize
  Se \textbf{Deposit} non è di tipo city, \textbf{Address} non può essere contemporaneamente in relazione con \textbf{Account} e \textbf{Deposit}. 
  }\\

  \textbf{isDriverAssignedToValidTransport} & Interrelazionale &
  {\footnotesize
  Il \textbf{Transport} associato a un \textbf{Driver} deve appartenere al \textbf{Deposit} in cui lavora.
  }\\

  \textbf{isShipmentAssignedToValidTransport} & Interrelazionale &
  {\footnotesize

  Il \textbf{Transport } associato a \textbf{Shipment} deve appartenere al \textbf{Deposit} di partenza della spedizione.
  }\\
}

\customTable{ccY}[Dizionario dei vincoli Quinta Parte]{\textbfB{Vincolo} & \textbfB{Tipologia} & \textbfB{Descrizione}}{

  \textbf{isShipmentDirectedInCorrectArea} & Interrelazionale &
  {\footnotesize

  Il \textbf{Transport} associato a una \textbf{Shipment} deve coprire la \textbf{Area} di destinazione della spedizione in data \textbf{ShippingDate}.
  La \textbf{Area} di destinazione di una \textbf{Shipment} è la \textbf{Area} dell'\textbf{Address} di un \textbf{Deposit} se la spedizione è diretta a un \textbf{Deposit}, altrimenti è la \textbf{Area} dell'\textbf{Address} di tutti gli \textbf{Account} destinatari.
  
  }\\

  \textbf{isShipmentContainingValidOrders} & Interrelazionale &
  {\footnotesize 
  Gli \textbf{Order} associati a una \textbf{Shipment} devono contenere \textbf{Product} che sono conservati nel \textbf{Deposit} di partenza della spedizione.
  }\\

  \textbf{formatOperatorBusinessmail} & Interrelazionale &
  {\footnotesize
  Il campo \textbf{Businessmail} della classe Operator deve avere il formato `\textbf{n}.\textbf{surnameN}@uninadelivery.operator.com', dove \textit{n} è l'iniziale del nome dell'operatore, \textit{surname} è il suo cognome, mentre \textit{N} è un numero intero positivo, da utilizzare per eventuali omonimie.
  }\\
  
  \textbf{formatDriverBusinessmail} & Interrelazionale &
  {\footnotesize
  Il campo \textbf{Businessmail} della classe Driver deve avere il formato `\textbf{n}.\textbf{surnameN}@uninadelivery.driver.com', dove \textit{n} è l'iniziale del nome dell'autista, \textit{surname} è il suo cognome, mentre \textit{N} è un numero intero positivo, da utilizzare per eventuali omonimie.
  }\\

  \textbf{isAccountReachable} & Interrelazionale &
  {\footnotesize
  Un \textbf{Account} deve avere almeno un \textbf{Deposit} nella stessa \textbf{City}. Le \textbf{City} in cui non è presente un \textbf{Deposit} sono da considerare non ancora raggiungibili dal servizio.
  }\\

  \textbf{isCityDepositShippingToClient} & Interrelazionale &
  {\footnotesize
  Se una \textbf{Shipment} è diretta a un \textbf{Account}, il \textbf{Deposit} di partenza della spedizione deve essere nella stessa \textbf{City}.
  }\\

}

\customTable{ccY}[Dizionario dei vincoli Sesta Parte]{\textbfB{Vincolo} & \textbfB{Tipologia} & \textbfB{Descrizione}}{
    
  \textbf{correctTransportForDrivingLicence} & Interrelazionale &
  {\footnotesize
  Un \textbf{Driver} con \textbf{DrivingLicenceType} uguale a \textbf{BE} può guidare solo \textbf{Transport} di tipo \textbf{WheeledSmall}, mentre se di tipo \textbf{CE} può guidare anche \textbf{Transport} di tipo \textbf{WheeledLarge}. Un \textbf{Driver} con \textbf{DrivingLicenceType} uguale a \textbf{DE} può guidare solo \textbf{Transport} di tipo \textbf{WheeledLarge}.
  }\\

  \textbf{correctTransportAssignment} & Interrelazionale &
  {\footnotesize
  Un \textbf{Transport} di tipo \textbf{WheeledSmall} può appartenere a ogni tipo di deposito, uno di tipo \textbf{WheeledLarge} non può appartenre a depositi di tipo \textbf{City}, uno di tipo \textbf{Rails} non può appartenere a depositi di tipo \textbf{City} o \textbf{State}, mentre quelli di tipo \textbf{Air} e \textbf{Water} possono solo appartenre a depositi di tipo \textbf{Central}.
  }\\

  \textbf{onlyOneJobPerAccount} & Interrelazionale &
  {\footnotesize
  Un \textbf{Account} non può essere contemporaneamente un \textbf{Driver} e un \textbf{Operator}.
  }\\

  \textbf{correctOccupiedSpaceForShipment} & Interrelazionale &
  {\footnotesize
  Il campo \textbf{OccupiedSpace} di un \textbf{Covers} deve essere maggiore o uguale alla somma dei campi \textbf{PackageSizeLiters} dei \textbf{Product} contenuti negli \textbf{Order} associati alla \textbf{Shipment} che lo trasporta.
  }\\

}